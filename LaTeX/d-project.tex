%%%%%%%%%%%%%%%%%%%%%%%%%%%%%%%%%%%%%%%%%%%%%%%
% These are the general sections to include.  %
%                                             %
% You can alter some names, but follow the    %
% suggestions in the NSF guidelines.          %
%                                             %
% If spacing is tight, play with negative     %
% vspaces w/in the text to reduce whitespace. %
%%%%%%%%%%%%%%%%%%%%%%%%%%%%%%%%%%%%%%%%%%%%%%%

\subsection{Context of OSE}

Describing the context and vision of the proposed OSE. This required section must include a description of the 
\begin{itemize}
    \item guiding principles and long-term vision for the proposed OSE,
    \item the specific societal or national need(s) that the OSE will address,
    \item the anticipated broader impacts of the OSE 
\end{itemize}


The section must have:
\begin{enumerate}
\item a pointer to the existing publicly-available open-source product that is being transitioned;
\item details on the current status of the research product, development model, methods of dissemination, and user base; and
\item a description of the problem being addressed, and
\item the novelty of the intended product being transitioned, including 
\begin{enumerate}
\item substantiating evidence of the technology's potential to significantly impact/address the problem.
\end{enumerate}
\end{enumerate}


\subsection{Ecosystem Establishment/Growth:}
  - Include a well-developed ecosystem` establishment/growth` and `ongoing discovery strategy` that ensures that the proposed OSE will:
  
    - further `develop the open-source product within the current technological landscape`, along with
  
  - specific plans to identify, engage and support potential users and partners who will serve as early adopters for the product`; 
  
  - specific plans to engage industrial and international collaborators are encouraged; 

\subsection{Organization and Governance: }
  - Describe a well-developed and sustainable `organizational`, `coordination`, and `governance` model including :
  
    - the licensing approach to be employed, 
    
    - the specific continuous development, integration and deployment processes and 
    
    - infrastructure that will enable the open, asynchronous, and distributed development of the open-source product and support sustainability goals for the OSE, along with 
    
    - metrics to assess and evaluate success, in the longer term, of the development methodology and 
    
    - processes for ensuring quality control, 
    
    - security and
    
    - privacy of new content;
    
\subsection{Community Building: }
- Describe a long-term strategy for community building to `engage`, `incentivize`, and `onboard` potential content contributors who will help in further developing and maintaining the open-source product;

\subsection{Sustainability: }
- Articulate clear sustainability goals of the OSE, and 

  - an actionable evaluation plan, along with 
    - metrics to assess and evaluate success, in the longer term, of the development methodology, processes for ensuring quality control, security and privacy of new content, support for users, and onboarding mechanisms for new contributors.



%%%%%%%%%%%%%%%%%%%%%%%%%%%%%%
% Section 4: Management Plan %
%%%%%%%%%%%%%%%%%%%%%%%%%%%%%%
\section{Time Line and Management Plan}

\begin{table}[H]
\label{table1}
\renewcommand{\arraystretch}{0}
\caption{Project schedule.  PIs are Person One (P1), Person Two (P2), graduate student is GS, and the undergraduate student is US.\ Time frame gives the year each activity will occur.}
\scriptsize
\begin{tabularx}{\textwidth}{Y c c }
\hline
\hline
\textbf{Research Activity} & \textbf{Personnel} & \textbf{Time Frame}\\
\hline
Perform a task that sounds impressive & P2, US & Y1 \T\\
Perform another super-amazing task & P1, US & Y1 \T\\
Perform something else that may not be as sexy as the other things & P2, GS & Y1 \T\\
Wonder why you are such a terrible programmer & P1, US & Y1 \T\\
Analyze the results and stuff & P1, P2, SS & Y1,Y2 \T\\
Take the day off and grill some meat & P1, P2, SS & Y1,Y2 \T\\
Present findings at scientific meetings and publish results in peer-reviewed journals & P1, P2, US, GS & Y1, Y2, Y3\T\B\\
\hline
\hline
\end{tabularx}
\end{table}

%%%%%%%%%%%%%%%%%%%%%%%%%%%%%%
% Section 5: Science Merit   %
%%%%%%%%%%%%%%%%%%%%%%%%%%%%%%
\section{Scientific Merit}

Stuff and things that will make new knowledge and make science better


%%%%%%%%%%%%%%%%%%%%%%%%%%%%%%
% Section 6: Impact/Outreach %
%%%%%%%%%%%%%%%%%%%%%%%%%%%%%%

\section{Broader Impacts}
\label{broadimpacts}
\vspace*{-8pt}


\vspace{4pt}
\noindent \underline{\textit{Data Access}}: Maybe write about you will make data available.

\vspace{4pt}
\noindent \underline{\textit{Student Training}}: Write about how you will train students.

\vspace{4pt}
\noindent \underline{\textit{Some Other Outreach}}: Write about more outreach.

\vspace{4pt}
\noindent \underline{\textit{Dissemination}}: Write about how you will disseminate results (i.e., journal articles, workshops, etc).

%%%%%%%%%%%%%%%%%%%%%%%%%%%%%%
% Section 7: Prior NSF Work  %
%%%%%%%%%%%%%%%%%%%%%%%%%%%%%%
\section{Results from Prior NSF Support}

\noindent \emph{\underline{Person One}}: No NSF support in the past five years \newline


\nocite*