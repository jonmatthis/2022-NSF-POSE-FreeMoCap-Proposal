\noindent {\bf Project Summary:}

\noindent This proposal seeks resources to develop a healthy and robust open source ecosystem to organize and support the long term growth and stability of the nascent community of scientists, artists, and open source developers that is forming around the \textbf{FreeMoCap} Software - A free, open source markerless motion capture system capable of producing research grade 3d kinematic data using generic, minimal cost camera hardware and freely available open source software. 

 Such an ecosystem will facilitate the long term stability, support, and growth of a vibrant, inclusive community dedicated to continuous development and support of high-quality, minimal cost motion capture software. Properly enacted, such a community could have a transformative effect on the various communities of researchers and artists who stand to benefit from it.
 
As the `freemocap' software stabilizes and a community of users and contributors begins to form around it, new challenge it becomes incumbent upon the core maintainers (represented as named parties in this proposal) to organize the \textbf{social, technical, and educational} infrastructure around this project:

\begin{itemize}

    \item \textbf{Social:} Support, maintain, and develop the growing community of freemocap users using various mechanisms for social engagement and communication (e.g. by increasing the sophistication of our current uses of Discord, Twitter, Twitch, Youtube, etc)
    
    \item \textbf{Technical:} Solidify the software infrastructure to facilitate long-term support and external contributions by implementing widely accepted professional standard for open source software development (e.g. Continuous integration, automated testing, and Github based project management, etc)
    
    \item \textbf{Educational:} Develop solid knowledge base with a focus on welcoming non-technically trained newcomers as well building bridges between the constituent sub-communities of specialized expert users (e.g. building documentation , tutorials, etc using the \textit{diataxis} framework) bn  

\end{itemize}

\noindent {\bf \large Intellectual Merit:}

\noindent The rapid advance of machine learning techniques generates wildly impressive results on a near-daily basis, but these techniques are often poorly suited for rigorous scientific research or professional use. The FreeMoCap software combines state-of-the-art, machine-learning driven, computer vision based markerless tracking methods with classical techniques from computational geometry to create an easy-to-use system for accurately recording the full-body 3d kinematics of human, animal, and robotic subjects for orders of magnitude lower cost than traditional marker-based systems. 

\noindent {\bf \large Broader Impacts: }

A high-quality, minimal-cost motion capture system would be a transformative tool for a wide range of communities - including 3d animators, game designers, athletes, coaches, performers, scientists, engineers, clinicians, and doctors. To ensure the that this tool is of use to the widest possible audience, we follow a `Universal Design' development philosophy, with the goal of creating \textbf{a system that serves the needs of a professional research scientist while remaining intuitive to a 13-year-old with no technical training and no outside assistance.}

By creating an ecosystem by which this tool is maintained both in service of specialized professional researchers as well as artists, athletes, and other non-technical users, we stand to create meaningful, productive connections between the forefronts of science and technology and communities that are traditionally excluded from them. A healthy and robust open source ecosystem organized around this central mission could have an unprecidented impact on the landscape of human-centered research and artistic expression. 